\documentclass[a4paper,12pt,oneside,fleqn]{article}
\usepackage[latin1]{inputenc}
\usepackage[T1]{fontenc}
\usepackage[finnish, english]{babel}
\usepackage{times}
\usepackage[pdftex]{graphicx}
\usepackage[left=35mm,right=20mm,top=20mm,bottom=25mm]{geometry}
\usepackage{setspace}
\usepackage{sectsty}
\usepackage{fancyhdr}
\usepackage{icomma}
\usepackage{amssymb,amsmath}
\usepackage[titles]{tocloft}
\usepackage{ccaption}
\usepackage{natbib}
\usepackage{colortbl}
\usepackage[none]{hyphenat}
\sloppy

\graphicspath{{figures/}}

\setlength{\headheight}{10mm}
\pagestyle{fancy}
\renewcommand{\headrulewidth}{0pt}
\renewcommand{\footrulewidth}{0pt}
\singlespacing
\newenvironment{equation_text}[0]
{\begin{list}{}{\setlength{\leftmargin}{25mm}}}
{\end{list}}

\newcommand{\researchname}[0]{Part 2}

\newcommand{\flabel}[1]{\label{fig:#1}}
\newcommand{\tlabel}[1]{\label{tab:#1}}
\newcommand{\elabel}[1]{\label{eq:#1}}
\newcommand{\slabel}[1]{\label{sec:#1}}

\newcommand{\fref}[1]{Figure~\ref{fig:#1}}
\newcommand{\tref}[1]{Table~\ref{tab:#1}}
\newcommand{\eref}[1]{Equation~\ref{eq:#1}}
\newcommand{\sref}[1]{Section~\ref{sec:#1}}

\newcommand{\tcaption}[1]{\caption{#1}\vspace{3mm}}

\newenvironment{enu}
{\begin{enumerate}
	\setlength{\topsep}{0pt}
  \setlength{\itemsep}{0pt}
  \setlength{\partopsep}{0pt}
  \setlength{\parskip}{0pt}
  \setlength{\parsep}{0pt}}
{\end{enumerate}}

\newenvironment{ite}
{\begin{itemize}
	\setlength{\topsep}{0pt}
  \setlength{\itemsep}{0pt}
  \setlength{\partopsep}{0pt}
  \setlength{\parskip}{0pt}
  \setlength{\parsep}{0pt}}
{\end{itemize}}

\renewcommand{\cftsecfont}{\normalfont}
\renewcommand{\cftsecpagefont}{\normalfont}
\renewcommand{\cftsecleader}{\cftdotfill{\cftdotsep}}
\setcounter{tocdepth}{3}
\setlength{\parskip}{0mm}
\setlength{\parindent}{0mm}
\setlength{\mathindent}{15mm}

\sectionfont{\fontfamily{phv}\fontseries{b}\fontsize{16pt}{18pt}\selectfont\uppercase}
\subsectionfont{\fontfamily{phv}\fontseries{b}\fontsize{14pt}{16pt}\selectfont}
\subsubsectionfont{\fontfamily{phv}\fontseries{m}\fontsize{12pt}{16pt}\selectfont}

\captiondelim{. }
\captionstyle{\raggedright}
\precaption{\hspace{5mm}}
\captiontitlefont{\fontfamily{phv}\fontshape{it}\fontsize{11pt}{12pt}\selectfont}
\bibpunct{/}{/}{,}{n}{}{,}

\definecolor{lightgrey}{rgb}{.8,.8,.8}
\definecolor{lightlightgrey}{rgb}{.9,.9,.9}
\definecolor{darkgrey}{rgb}{.5,.5,.5}
\definecolor{darkdarkgrey}{rgb}{.4,.4,.4}

\begin{document}

\renewcommand{\figurename}{\fontfamily{phv}\fontseries{b}\fontsize{11pt}{12pt}\selectfont Figure}
\renewcommand{\tablename}{\fontfamily{phv}\fontseries{b}\fontsize{11pt}{12pt}\selectfont Table}

% TITLE PAGE
\thispagestyle{empty}
\addtolength{\hoffset}{-5mm}
\begin{flushleft}
{\fontfamily{phv}\selectfont\textbf{EINDHOVEN UNIVERSITY OF TECHNOLOGY}} \\
{\fontfamily{phv}\selectfont\textbf{Department of Mathematics and Computer Science}} \\
{\fontfamily{phv}\selectfont\textbf{Department of Computer Science}} \\
{\fontfamily{phv}\selectfont\textbf{2ID35 Database Technology}} \\
{\fontfamily{phv}\selectfont\textbf{Project}} \\
\end{flushleft}

\vfill

\begin{center}
{\fontfamily{phv}\selectfont\textbf{\researchname}} \\
%{\fontfamily{phv}\selectfont Supervisor: Title Supervisor name} \\
\end{center}

\vfill

\begin{flushright}
\begin{tabular}{l}
{\fontfamily{phv}\selectfont In Eindhoven} \\
{\fontfamily{phv}\selectfont 30.03.2010} \\
{\fontfamily{phv}\selectfont Puputti Kimmo (0735552), Geert Kemps (0514520)} \\
%{\fontfamily{phv}\selectfont student number} \\
{\fontfamily{phv}\selectfont k.p.puputti@student.tue.nl, g.c.m.kemps.1@student.tue.nl }
\end{tabular}
\end{flushright}
\clearpage

% SECTIONS
\pagestyle{fancy}
\fancyhf{}
\rhead{\thepage}
\setcounter{page}{1}
\setstretch{1.1}

\section{Motivation}



\section{Research problems in the article}

Traditionally service providers want to hold on to their data as long
as possible. The data, after all, is the most valuable asset to
them. However, with a lot of detailed and personal information about
the users in a database, the users of the service often lose their
privacy over the needs of the service providers. With a lot of
personal data and the commonness of data breaches, the data can get
into wrong hands if saved long enough.\\

Normally companies have some fixed time that information is saved, and
after the data is old enough, it is removed completely. This presents
many problems, since the data is valuable to the providers, but the
removal of data is important for respecting the privacy of the
users.\\

With the service providers' and users' needs being the opposite of
each other, a better was to save and remove data is needed. Van Heerde
et al. propose a way to degrade the data over time, so that not all
information is kept over time, but some parts of it is removed. This
tries to be a compromise between the needs of the providers and the
needs of the users.\\

By keeping some data around longer, the service providers can still
use the valuable data for their needs, but the privacy of the users is
better than in the case where all data is kept. With the degraded
data, the information about the users is not so accurate anymore, and
possible data breaches don't affect the users that much anymore.\\

Van Heerde et al. present a framework for degrading the data over
time. It tries to optimize the common interest of the service
providers and the users by balancing privacy and data usability. The
framework allows to compute the interest of the service providers and
the interest of the users separately, and offers a way to combine
these to a common interest.

\section{Results in the article}

The framework for data degradation described in the article by Van
Heerde et al. used two functions for computing the common interest of
the service providers and the users. First function computes the worth
of storing the data for the service provider, and the second one
computes the risk for involved for storing the data.\\

With the defined functions, Van Heerde et al. were able to find the
optimal retention period so that the common interest of the both
parties was respected.\\

The common interest of the two parties depends on many
factors. Sometimes the decrease of worth is higher than the decrease
in privacy. Van Heerde et al. showed that progressively degrading the
data is useful in the cases they presented.

\section{Plan for verifying the claims}

For evaluating the results of the article, we are going to use the
proposed methods for data degradation for a part of the query log data
set released by AOL a few years ago (
http://techcrunch.com/2006/08/06/aol-proudly-releases-massive-amounts-of-user-search-data/
).\\

The query log contains 36 million records with five fields: id of the
user, query string, time of the query, ranking of the clicked link,
and the clicked link itself. The last two fields can be empty.\\

The data set is analyzed using Python programming language. It some
plots are needed for visualizing the results, Matplotlib Python
plotting library ( http://matplotlib.sourceforge.net/ ) is used.\\

We will provide means for calculating the worth of the data and the
risk of the data automatically. We will then find the optimal common
interest for the data and try to degrade it using the functions
created. Then we will compare the results of the different levels of
degradation to get comparable results.

\end{document}
